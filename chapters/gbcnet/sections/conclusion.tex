\section{Conclusion}
%\todo{We have observed that our proposed system fails to capture the inaccuracies caused by the small mass or irregular structures near the gallbladder wall. Since gallbladder is a hollow organ, the malignancy starts from the wall. For nearly 5\% of the images, the irregularity and minuscule structures of the wall needs discerning between the noise, soft tissue texture, and the tiny wall structure.} 
In this chapter, we addressed Gallbladder Cancer detection from USG images using deep learning and proposed a new supervised learning framework (GBCNet) based on \roi selection and a multi-scale second-order pooling. The proposed design helps the classifier focus on the crucial \gb region predicted by the region selection network, and make better predictions by mitigating the effect of noise and artifacts. We also tested the performance of the proposed MS-SoP classifier on breast cancer detection from USG. The performance gain using MS-SoP on these two different types of cancers indicates that the model is capable of learning robust representations of different types of malignancy in USG images. We also proposed a visual acuity-based curriculum to make our design resilient to texture bias and improve its specificity of GBCNet. Extensive experiments show that GBCNet, combined with the curriculum learning, improves performance over the baseline deep classification and object detection architectures. Our work tackles the impediments posed by \usg images towards making accurate GBC detection, an important but hitherto overlooked problem.
%In the future, we would like to explore if the proposed network architecture and the visual acuity-based curriculum can help improve predictions in other cancer detection problems.
%We chose breast cancer detection, a prominent problem in this domain, and validated the MS-SoP classifier on the public breast ultrasound image (BUSI) dataset. We note that while breast cancer detection relies on tissue/ mass characterization, GBC detection is primarily based on wall shape and mass anomaly. The performance gain using MS-SoP on these two different types of cancers indicates efficient malignancy representation by our model. We plan to evaluate our method on other datasets as future work to formally understand the efficacy and the characterization of malignancy representation by MS-SoP for different types of cancers.

%{\small \mypara{Acknowledgement} The authors thank IIT Delhi HPC facility for computational resources.}